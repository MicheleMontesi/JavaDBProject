\documentclass[a4paper, 12pt]{report}

\usepackage{alltt, fancyvrb, url}
\usepackage{graphicx}
\usepackage[utf8]{inputenc}
\usepackage{float}
\usepackage{hyperref}
\usepackage{xurl}
\usepackage{lipsum}  

\usepackage[italian]{babel}
\usepackage[italian]{cleveref}
\usepackage[toc,page]{appendix}

\newenvironment{sloppypar*}{\sloppy\ignorespaces}{\par}

\newenvironment{changemargin}[2]{%
  \begin{list}{}{%
    \setlength{\topsep}{0pt}%
    \setlength{\leftmargin}{#1}%
    \setlength{\rightmargin}{#2}%
    \setlength{\listparindent}{\parindent}%
    \setlength{\itemindent}{\parindent}%
    \setlength{\parsep}{\parskip}%
  }%
  \item[]}{\end{list}}

\title{Relazione Progetto Basi di Dati 2022}
\author{Michele Montesi \\
        Matricola: 0000974934 \\
        E-Mail: michele.montesi3@studio.unibo.it}

\date{\today}

\begin{document}
    
\maketitle
\tableofcontents

\chapter{Analisi Dei Requisiti}
Si vuole realizzare un database a supporto dell'automatizzazione della gestione di una
comunitá la quale gestisce diverse residenze per pazienti psichiatrici.
La base di dati dovrá immagazzinare informazioni relative agli operatori, ai pazienti e
alle varie residenze.

\section{Intervista}
Un primo testo ottenuto dall'intervista è il seguente:

\begin{changemargin}{1cm}{1cm}
        Si vuole tenere traccia di \textbf{pazienti} e \textbf{dipendenti} memorizzandone le informazioni personali
        quali codice fiscale, nome, cognome, compleanno e sesso. Differenziando le due entitá
        si memorizzeranno, inoltre, le informazioni legate alla propria posizione.
        Questi due soggetti possono essere registrati nel sistema solo accettando alcune clausole o
        disponendo di eventuali requisiti.\\
        I \textbf{dipendenti} possono acquisire degli attestati che gli conferiscono crediti ECM e possono firmare
        contratti (prima di firmare un secondo contratto con lo stesso nome, il primo deve essere concluso).\\
        I \textbf{turni} dei dipendenti sono determinati dal codice fiscale del dipendente, il giorno della settimana, 
        l'ora di inizio, l'ora di fine e l'unitá operativa in cui si svolgeranno.\\
        I \textbf{pazienti} sono identificati da una \textbf{cartella clinica} la quale contiene informazioni riguardanti
        l'anamnesi, la diagnosi e il progetto riabilitativo del paziente.\\
        Sono presenti \textbf{unitá operative} (le quali possono essere \texttt{gruppi appartamento} o \texttt{residenze
        sanitarie psichiatriche}) adibite all'\textbf{ospitazione} di piú pazienti (la quale dovrá essere registrata 
        con data di inizio e opzionalmente con una data di fine). Queste sono caratterizzate dalla loro
        ubicazione, i posti letto e il numero dei pazienti. Per poter essere operative devono ricevere
        un'\texttt{autorizzazione al funzionamento} e l'\texttt{accreditamento}.
\end{changemargin}

\section{Estrazione dei concetti principali}

\chapter{Progettazione Concettuale}

\section{Schema scheletro}

\section{Schema finale}

\chapter{Progettazione Logica}

\section{Stima del volume dei dati}


\end{document}