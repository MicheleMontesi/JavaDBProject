\documentclass[a4paper, 12pt]{report}

\usepackage{alltt, fancyvrb, url}
\usepackage{graphicx}
\usepackage[margin=2cm]{geometry}
\usepackage[utf8]{inputenc}
\usepackage{float}
\usepackage{hyperref}
\usepackage{xurl}
\usepackage{lipsum}
\usepackage[htt]{hyphenat}
\usepackage{tabularx}
\usepackage{array}
\usepackage{color}
\usepackage{colortbl}
\usepackage[table]{xcolor}

\usepackage[italian]{babel}
\usepackage[italian]{cleveref}
\usepackage[toc,page]{appendix}

\definecolor{seaGreen}{cmyk}{0.19, 0, 0.25, 0.09}

\newenvironment{sloppypar*}{\sloppy\ignorespaces}{\par}

\newenvironment{changemargin}[2]{%
  \begin{list}{}{%
    \setlength{\topsep}{0pt}%
    \setlength{\leftmargin}{#1}%
    \setlength{\rightmargin}{#2}%
    \setlength{\listparindent}{\parindent}%
    \setlength{\itemindent}{\parindent}%
    \setlength{\parsep}{\parskip}%
  }%
  \item[]}{\end{list}}

\title{Relazione Progetto Basi di Dati 2022}
\author{Michele Montesi \\
        Matricola: 0000974934 \\
        E-Mail: michele.montesi3@studio.unibo.it}

\date{\today}

\begin{document}
    
\maketitle
\tableofcontents

\chapter{Analisi Dei Requisiti}
Si vuole realizzare un database a supporto dell'automatizzazione della gestione di una
comunità la quale gestisce diverse residenze per pazienti psichiatrici.
La base di dati dovrà immagazzinare informazioni relative agli operatori, ai pazienti e
alle varie residenze.

\section{Intervista}
Un primo testo ottenuto dall'intervista è il seguente:

\begin{changemargin}{0.5cm}{0.5cm}
        \noindent
        Si vuole tenere traccia di \textbf{pazienti} e \textbf{dipendenti} memorizzandone le informazioni personali
        quali codice fiscale, nome, cognome, compleanno e sesso.
        Differenziando le due entità si memorizzeranno, inoltre, le informazioni legate alla propria posizione.
        Questi due soggetti possono essere registrati nel sistema solo accettando alcune clausole o
        disponendo di eventuali requisiti.\\
        I \textbf{dipendenti} possono acquisire degli \textbf{attestati} che gli conferiscono crediti ECM e possono firmare
        \textbf{contratti} (prima di firmare un secondo contratto con lo stesso nome, il primo deve essere concluso).
        Di entrambi verrà mantenuto uno storico.\\
        I \textbf{turni} dei dipendenti sono determinati dal codice fiscale del dipendente, il giorno della settimana, 
        l'ora d'inizio, l'ora di fine e l'unità operativa in cui si svolgeranno.\\
        I \textbf{pazienti} sono identificati da una \textbf{cartella clinica} la quale contiene informazioni riguardanti
        l'anamnesi, la diagnosi e il progetto riabilitativo del paziente. A ognuno di questi verrà assegnata una 
        \textbf{terapia} la quale dovrà essere seguita assumendo farmaci (somministrati da dipendenti) e della quale
        verrà mantenuto uno storico.\\
        I \textbf{farmaci} sono caratterizzati dal loro codice, il nome, la casa farmaceutica, le date d'acquisto e
        di scadenza e la quantità.\\
        Sono presenti \textbf{unità operative} (le quali possono essere \texttt{gruppi appartamento} o \texttt{residenze
        sanitarie psichiatriche}) adibite all' \textbf{ospitazione} di più pazienti (la quale dovrà essere registrata
        con data d'inizio e opzionalmente con una data di fine) della quale verrà mantenuto uno storico.
        Queste sono caratterizzate dalla loro ubicazione, i posti letto e il numero dei pazienti.
        Per poter essere operative devono ricevere l'\texttt{autorizzazione al funzionamento} e l'\texttt{accreditamento}.
        Ogni unità operativa avrà una lista di \textbf{beni strumentali}, i quali possono essere automezzi o attrezzature.
\end{changemargin}

\section{Estrazione dei concetti principali}\label{sec:estrazione-dei-concetti-principali}
\begin{tabularx}{\textwidth}{lXc}
        \rowcolor{seaGreen}
        \textbf{Termine} & \textbf{Breve descrizione} & \textbf{Eventuali sinonimi} \\
        Dipendente & Colui che è assunto dalla società e lavora in una o più strutture a seconda dei turni e
        somministra farmaci ai pazienti.\ Può essere socio.\ & Lavoratore \\
        \hline
        Contratto & Oggetto contenente le ore lavorative mensili del dipendente a cui viene sottoscritto.\ & Contratto Lavorativo \\
        \hline
        Attestato & Oggetto che conferisce, al dipendente che ne consegue il completamento, crediti EMC (crediti formativi).\ & Formazione \\
        \hline
        Paziente & Colui che riceve le cure attraverso la somministrazione di farmaci, seguendo una terapia, ed è ospitato
        all'interno di una unità operativa.\ & Cliente \\
        \hline
        Cartella clinica & Documentazione del paziente contenente anamnesi, diagnosi e progetto riabilitativo.\ & Cartella \\
        \hline
        Terapia & Oggetto che riporta i farmaci da assumere durante un periodo di tempo.\ & Cura \\
        \hline
        Farmaco & Medicinale adibito all'assunzione da parte di pazienti a cui è stato assegnato.\ & Medicinale \\
        \hline
        Unità operativa & Residenza o appartamento i cui i pazienti risiedono e ricevono le cure da parte del personale.\ & Residenza, Unità \\
        \hline
        Bene strumentale & Strumento o veicolo aziendale necessario o utile a semplificare il lavoro o la permanenza nell'unità operativa.\ & Strumentazione, Veicolo
\end{tabularx}
\\\\\\
A seguito della lettura e comprensione dei requisiti, si procede redigendo un testo che ne 
riassuma tutti i concetti e in particolare ne estragga quelli principali eliminando le ambiguità 
sopra rilevate:

\begin{changemargin}{0.5cm}{0.5cm}
        Per ogni \textbf{Dipendente} 
\end{changemargin}

\chapter{Progettazione Concettuale}

\section{Schema scheletro}

\section{Schema finale}

\chapter{Progettazione Logica}

\section{Stima del volume dei dati}


\end{document}