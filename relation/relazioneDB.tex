\documentclass{report}

\usepackage{alltt, fancyvrb, url}
\usepackage{graphicx}
\usepackage[utf8]{inputenc}
\usepackage{float}
\usepackage{hyperref}
\usepackage{xurl}
\usepackage{lipsum}  

\usepackage[italian]{babel}
\usepackage[italian]{cleveref}
\usepackage[toc,page]{appendix}

\title{Relazione Progetto Basi di Dati 2022}
\author{Michele Montesi \\
        Matricola: 0000974934 \\
        E-Mail: michele.montesi3@studio.unibo.it}

\date{\today}

\begin{document}
    
\maketitle
\tableofcontents

\chapter{Analisi Dei Requisiti}
Si vuole realizzare un database a supporto dell'automatizzazione della gestione di una
comunitá la quale gestisce diverse residenze per pazienti psichiatrici.
La base di dati dovrá immagazzinare informazioni relative agli operatori, ai pazienti e
alle varie residenze.

\section{Intervista}
Un primo testo ottenuto dall'intervista é il seguente:
\newline
Si vuole tenere traccia di pazienti e dipendenti memorizzandone le informazioni personali
quali codice fiscale, nome, cognome, compleanno e sesso. Differenziando le due entitá
si memorizzeranno, inoltre, le informazioni legate alla propria posizione. 

\section{Estrazione dei concetti principali}

\chapter{Progettazione Concettuale}

\section{Schema scheletro}

\section{Schema finale}

\chapter{Progettazione Logica}

\section{Stima del volume dei dati}


\end{document}